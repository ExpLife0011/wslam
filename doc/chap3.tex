\chapter{Used technologies}
    \section{File System Legacy Filter and Minifilter Drivers}
        \paragraph{}
        A Legacy Filter driver is a kernel-mode that could attach to a device's stack. In the context of file system filtering, these
        filter drivers could intercept file system I/O operations. Developing legacy filter drivers was quite troublesome and led to 
        many incompatibilities between filter drivers. This is one of the reasons for which minifilter drivers were added.
        
        \paragraph{}
        Minifilters have the same abilities as file system legacy filter drivers, but they are easier to develop and are overall safer. Their
        load order no longer depends on the attach order, but on a predefined value named altitude. Minifilters are managed by FltMgr, which is
        a legacy filter driver implemented by Microsoft.

        \paragraph{}
        I have used this technology for the core component of the developed application, which had to filter file system operations as well
        as process creation and termination. All of this information can't be reliably be acquired from a user mode application, so a minifilter
        driver was the only choice.

    \section{C++ in Kernel Drivers}
        \paragraph{}
        C++ has the advantage of being easy to use in developing coherent, object-oriented, robust and safe applications. However, msvc compiler
        does not, by default, support c++ in kernel drivers.

        \paragraph{}
        Firstly, in order to support global initializers, we need to define two sections, ".CRT\$XCA" and ".CRT\$XCZ". Between these two
        sections all pointers to global initializers are held.

        \paragraph{}
        Secondly, In order to support static object initialization, we need iterate through the list of global initializers and call each
        constructor in the DriverEntry function, and call the destructors during driver unload.

        \paragraph{}
        Thirdly, we need to define the global new and delete operators.

        \paragraph{}
        Lastly, we need to implement an the \textunderscore purecall() which will be called whenever a pure virtual function is called. Microsoft's implementation
        of this function causes the program to immediately terminate if there is no exception handler for it.

    \section{.NET Framework}
        \paragraph{}
        I have used .NET Framework for the integration project, both for the system service and the GUI application. The GUI was developed with
        the help of Windows Presentation Foundation (WPF), with the interface designed using Extensible Application Markup Language (XAML), and
        the communication between the service and GUI was done through Windows Communication foundation (WCF), more exactly, through the
        net.tcp protocol. WCF was particulary useful as it exposes an easy to use, RPC-like comomunication interface

    \section{C++/CLI}
        \paragraph{}
        The C++ modified Common Language Infrastracutre is a c++ based language specification developed by Microsoft. It offers the possibility
        to use both the unmanaged CRT heap (allocating and freeing objects) as well as the .NET managed heap, where objects are garbage collected
        and don't have to be freed by the programmer. This technology was helpful because it allows an easy linkage with a C or C++ DLL and
        provides a simple way of wrapping the dll's functions and classes in order to export them as .NET objects to .NET application, which,
        in this case, was a service.



    
