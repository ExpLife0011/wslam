\chapter{Used technologies}
    In order to develop a robust and modern application, but also an efficient and reliable monitoring and detection system, I have used various
    technologies, ranging from low level kernel technologies to high level .NET frameworks.

    \section{Minifilter Drivers}
        A Legacy Filter driver is a kernel-mode driver that can attach to a device's stack. In the context of file system filtering, these
        filter drivers could intercept file system I/O operations. Developing legacy filter drivers was quite troublesome and led to 
        many incompatibilities between filter drivers. This is one of the reasons for which minifilter drivers were added.
        
        \paragraph{}
        Minifilters have the same abilities as file system legacy filter drivers, but they are easier to develop and are overall safer. Their
        load order no longer depends on the device stack attach order, but on a predefined value named altitude. Minifilters are managed by
        FltMgr, which is a legacy filter driver implemented by Microsoft.

        \paragraph{}
        I have used this technology for the core component of the developed SDK, which had to filter file system operations as well
        as process creation and termination. All of this information can't be reliably be acquired from a user mode application, so a minifilter
        driver was the only reliable choice.

    \section{C++ in Kernel Drivers}
        C++ has the advantage of being easy to use in developing coherent, object-oriented, robust and safe applications. However, msvc compiler
        does not, by default, support C++ in kernel drivers. Therefore, I had to implement C++ support from scratch.

        \paragraph{}
        Firstly, in order to support global initializers, we need to define two sections, ".CRT\$XCA" and ".CRT\$XCZ". All pointers to 
        initializers reside between these two sections.

        \paragraph{}
        Secondly, In order to support static object initialization, we need iterate through the list of global initializers and call each
        constructor in the DriverEntry function, and call the destructors during driver unload.

        \paragraph{}
        Thirdly, we need to define the global new and delete operators. However, since memroy allocations done in a kernel driver need to specify
        a memory pool type and a tag, I had also overloaded the new and delete operators, in order to support two additional parameters. This
        way, the new operator can be called as follows:

        % TODO - syntax highlight.

        \begin{Verbatim}
auto ptr = new(NonPagedPoolNx, 'GAT\#') Process{arg1, arg2};
        \end{Verbatim}

        This will allocate a Process object from the non executable non paged pool with the "\#TAG" allocation tag.


        \paragraph{}
        Lastly, we need to implement the \textunderscore purecall() function, which will be called whenever a pure virtual function is called.
        Microsoft's implementation of this function causes the program to immediately terminate if there is no exception handler for it.

    \section{.NET Framework}
        \paragraph{}
        I have used .NET Framework for the integration project, both for the system service and the GUI application. The GUI was developed with
        the help of Windows Presentation Foundation (WPF), with the interface designed using Extensible Application Markup Language (XAML), and
        the communication between the service and GUI was done through Windows Communication foundation (WCF), more exactly, through the
        net.tcp protocol. WCF was particulary useful as it exposes an easy to use, RPC-like comomunication interface

        \paragraph{}
        I decided to use .NET technology for the integration project because it was easier to design a good and easy to use GUI application and
        communicate with it through a RPC like interface with the help of WCF. Although, this caused a problem with integration the C++ DLL that
        provided the control API and notifications for the driver.

    \section{C++/CLI}
        \paragraph{}
        The C++ modified Common Language Infrastracutre (CLI) is a C++ based language specification developed by Microsoft. It offers the possibility
        to use both the unmanaged CRT heap (allocating and freeing objects) as well as the .NET managed heap, where objects are garbage collected
        and don't have to be freed by the programmer.
        
        \paragraph{}
        This technology was helpful because it allows an easy linkage with a C or C++ DLL and provides a simple way of wrapping the dll's
        functions and classes in order to export them as .NET objects to .NET application, which, in this case, was a service.
    
