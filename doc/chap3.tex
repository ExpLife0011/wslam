\chapter{Architecture}
    \section{High Level Overview}
        \paragraph{}
        The system is composed of 4 main components
        \begin{itemize}
            \item wslflt.sys
            \item wslcore.dll
            \item wslsvc.exe
            \item wslam.exe
        \end{itemize}
    \section{wslflt.sys}
        \paragraph{}
        Wslflt.sys is C++ minifilter driver that contains the requirued sensors for monitoring process' activity. It's key components are the
        process filter, file filter, event dispatcher, the communication framework and the detection heuristics.
        \subsection{File Filter}
            \paragraph{}
            The file filter's role is to filter disk I/O.
        \subsection{Process Filter}
            \paragraph{}
            The process filter is notified of process and thread creation and termination by the windows kernel.
        \subsection{Event Dispatcher}
        \subsection{Communication Framework}
            \paragraph{}
        \subsection{Heurisitcs}
        \paragraph{}
    \section{wslcore.dll}
        \paragraph{}
        This dll is an abstraction of the wslflt.sys driver. It contains the communication logic between kernel-mode and user-mode and exports
        multiple callbacks to an integrator. It's main purpose is to hide the filtering and detection logic and to provide an easy way to integrate
        the system in a complete security solution.
        \paragraph{}
    \section{wslsvc.exe}
        \paragraph{}
        Wslsvc.exe is Windows service that integrates the previously mentioned DLL.
    \section{wslam.exe}
        \paragraph{}
        Wslam.exe is the GUI component of the system. It provides the user with logs about processes' actions and notifies him about detections
        or other suspicios activity.
        \paragraph{}
        It is an UWP .NET application that communicates with wslsvc via TCP/IP
