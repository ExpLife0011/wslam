\chapter{Heuristics and Detection Algorithms}
    \section{Privilege Escalation Detection}
    \paragraph{}
    When an unprivilleged process needs to start a privilleged process, it sends a request to svchost, which shows the User Account Control pop-up,
    notifying the user that the process needs admin rights. If admin rights are granted by UAC, svchost starts the process. If a process already
    is elevated, any process spawned by it will also be elevated. We can easily see that, if wsl.exe is started with admin rights, all linux
    processes running in that WSL instance will be granted admin rights inside Windows, which would be disastrous consoidering the security
    of the computer.
    \paragraph{}
    This kind of exploit can be detected easily, while not very reliable, from a kernel driver, or, more reliable, by using memory introspection
    techniques from a hypervisor. We will cover only the first technique.\\
    We need to check in key moments that the process' token was not tampered with. Such key moments would be when accessing files in sensitive
    Windows paths.
    \section{File Infector Detection}
    \section{Ransomware Detection}