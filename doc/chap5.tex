\chapter{Implementation Details}
    \section{Process Filtering}
        \begin{lstlisting}
            PsSetCreateProcessNotifyRoutineEx2(a, b);
        \end{lstlisting}
            
    This is the most reliable way of monitoring and blocking potentially malicious linux applications. Unlike a user-mode hooking framework
    approach, using a windows driver provides mechanisms of monitoring that cannot be bypassed or tampered with by the monitored processes, making
    it a very reliable and secure approach.\\
    The best choice for achieving reliable file system i/o and process monitoring is a file system minifilter driver.
    
    \section{File System Filtering}
    Windows file system i/o requests are contained in IRPs (i/o request packets). The minifilter driver registers its callback with the
    filesystem by calling FltRegisterFilter and passing an array of callbacks. The main difference between filtering i/o requests issued by win32
    processes and linux processes is that, for linux processes, the RequestorMode is KernelMode, because IRPs are issued by the kernel. This means
    that unlike the usual filtering algorithm, which skips KernelRequests, now we must process and check that the Requestor PID is not system.

    \paragraph{}
    When filtering file system operations, we are mostly interested in 2 fields of the FLT\textunderscore CALLBACK\textunderscore DATA structure,
    while the FILE\textunderscore OBJECT is taken from the FLT\textunderscore RELATED\textunderscore OBJECTS structure.

    \begin{verbatim}
0: kd> dt fltmgr!_FLT_CALLBACK_DATA
...
0x10 Iopb         : Ptr64 _FLT_IO_PARAMETER_BLOCK
...
0x50 RequestorMode: Char

0: kd> dt fltmgr!_FLT_RELATED_OBJECTS
...
0x20 FileObject   : Ptr64 _FILE_OBJECT
...
    \end{verbatim}

    \paragraph{}
    

    \section{Process Monitoring}
        \paragraph{}
        Process monitorig refers to getting synchronous notifications on both process creation and process termination in order to keep track
        of the currently active processes and process hierarchy.

        \paragraph{}
        In order to receive these notifications, the driver must register a callback with PsSetCreateProcessNotifyRoutineEx2. This callback will
        be called for both win32 processes and pico processes. To identify a WSL process, the driver must call ZwQueryInformationProcess with
        SubsystemInformationTypeWSL information type on the process' handle and check that the subsystem type is SubsystemInformationTypeWSL.
    
    \section{Communication}

    \section{Detection Flow}

    \section{Scoring Engine}