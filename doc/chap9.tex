\chapter{Conclusion}
    The reasearch of Windows Subsystem for Linux revealed an attack surface that, even though has not been seen yet to be exploited in the
    wild by actual malware, is a security issue that needs to be addressed and anti-malware solutions should be updated to take the Linux
    subsystem into consideration.

    \paragraph{}
    I have addressed these issues by developing a complete product that can offer basic protection against the described security issues by
    making use of behavioral detection heuristic algorithms and by improving these algorithms.  The proposed solution, being based on a 
    minifilter driver as the core component which encapsulates the monitoring and detection logic, is, as I have proven, a reliable method
    to cover the new attack surface and defend against basic bashware.

    \paragraph{}
    Finally, the possible directions in which research can be continued in order to improve efficiency and detection provide a starting point
    for further research and innovation in strengthening the security of WSL, making it safer for the end user.