\chapter{Testing}
    \paragraph{}
    Rigorous testing, especially in the context of kernel-mode modules and security solutions, is essential in order to provide a stable,
    usable and efficient product. Combining white box and black box techniques together with test driven developemnt paradigms throughout
    the development of a product is crucial in order to offer a high quality product.
    
    \section{Whitebox Testing}
        \paragraph{}
        In order to use the Test Driven Development paradigm, a white box unit testing framework was required. The unit test project contains
        mock definitions of kernel functions (e.g. ExAllocatePoolWithTag), and unit tests for each class. Mocking kernel functions helped in
        achieving high coverage of the driver code in a user-mode environment, thus saving development time. Moreover, these unit tests can be
        easily configured to run at each solution build, making it a very reliable and easy to use continuous integration tool.
        
        \paragraph{}
        The other alternative was having another kernel-mode driver that would test the API exported by wslflt.sys. This is an unreliable
        and slow testing method because most bugs would cause a blue screen and can even corrupt the testing environment. Moreover, analysing
        kernel dumps in order to find bugs is a much slower process compared to analysing user-mode crashes or exceptions. Also, building a
        continuous integration system around this testing framework is a lot more complicated because it involves virtual machines,
        automatically applying OS images and re-applying them in case a BSOD occurs.

        \paragraph{}
        In the context of the minifilter driver, white box test cases were implemented in order to verify that the Windows kernel APIs are
        correctly. These verifications include checks for handle leaks and correct error status handling. Moreover, the driver's components
        (i.e. file filter, process collector) are tested individually to confirm that they work correctly. Also, generic framework classes like
        SharedPointer or LookasideObject are covered by unit tests as they are widely used throughout the project.
    
    \pagebreak

    \section{Blackbox testing}
        Blackbox testing, both manual and automated, is the main method of attesting a product's value, usefulness and correctness regarding
        the users' needs. I have used black box techniques in order to verify both functional (i.e. detection) and non-functional requirements
        (i.e. stability, security, reliability).

        In order to identify memory or handle leaks, potential deadlocks, incorrect IRP handling, code integrity violation and other potential
        bugs, all black box tests are run with driver verifier enabled for wslft.sys. Using the driver verifier in the testing process helped
        in building a more robust and stable driver.

        Most black box test cases were implemented as bash scripts running in WSL, but some more complex test cases were also implemented
        using powershell.

    \section{Static Code Analysis}
        Static code analysis is a technique used to find potential bugs in a software without actually running it. Most code analysis tools use
        either the source code itself or some form of intermediate form of the code, (i.e. object code).

        SAL is a source code annotation language developed by Microsoft for C and C++. Its purpose is to clarify the intent behind the code,
        locking behavior or function parameters while also enabling static code analysis.

        Below is an example of SAL annotated function:

    \begin{Verbatim}[fontsize=\small, commandchars=\\\{\}]
\textcolor{magenta}{_Must_inspect_result_}
\textcolor{blue}{void} * memcpy(  
    \textcolor{magenta}{_Out_writes_bytes_all_}(count) \textcolor{blue}{void} *dest,   
    \textcolor{magenta}{_In_reads_bytes_}(count) \textcolor{blue}{const void} *src,   
    \textcolor{magenta}{_In_} \textcolor{blue}{size_t} count  
);  
    \end{Verbatim}

        \textunderscore Must\textunderscore inspect\textunderscore result\textunderscore
        - indicates that the return value of the function must be checked

        \textunderscore Out\textunderscore writes\textunderscore bytes\textunderscore all\textunderscore (count)
        - indicates an output parameter where we write exactly "count" bytes

        \textunderscore Out\textunderscore read\textunderscore bytes\textunderscore (count)
        - indicates an input parameter from which we read exactly "count" bytes

        \paragraph{}
        During code analysis, all these annotations are checked and any anomaly (i.e. writing to an \textunderscore In\textunderscore {} parameter) is
        reported as a compilataion warning.

    \section{WHQL Testing}
        In order to release a driver to the market, it needs to be digitally signed by microsoft, and the only way of receiving the signature, it
        must pass the WHQL test suite. In the case of wslflt.sys, that would be the Filter Driver Test Suite. These complex test suite
        contains extensive checks for reparse point handling, I/O stress conditions, transactional I/O, code integrity checks and much more for
        multiple file systems.
