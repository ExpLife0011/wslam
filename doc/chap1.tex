\chapter{Introduction}
    With the release of Windows Subsystem for Linux (WSL) in the "Redstone 1" version of Windows 10, consisting of two new kernel-mode drivers,
    lxcore.sys and lxss.sys, which implement more than  200 linux system calls, a new service, LxssManager, and other user mode components,
    a new attack surface has been revealed, one that was not covered by monitoring tools untill recent, and doesn't seem to be covered by
    anti-virus vendors.

    WSL provides a way of executing native 64 bit Linux binaries, also known as ELF64 files, on Windows 10. Since WSL is not virtualization
    based, the Linux processes running in WSL can access the same resources as the Windows processes, while not being monitored by existent
    tools or anti-malware solutions.

    \section{Motivation}
        Seeing that currently no anti-virus vendor tried to tackle this issue and develop a security solution that could cover this new
        complex attack surface, I have decided to do more research about WSL and try to come up with a full stack security solution that covers
        this attack surface.
        
        Moreover, seeing a proof of concept of a privilege escalation exploit I had another reason to think that developing a security solution
        for WSL was imperative.

    \section{Objectives}
        The main objective is to provide a partial solution to this new security problem. It should defend a system from bashware,
        suspicious interaction between WSL processes and Windows processes and to provide system administrators with enough logs to properly
        respond to incidents, while not disrupting the user experience. The monitoring and detection system had to be self contained in
        a software development kit (SDK), in order to be easily integrable by any third party AV vendors.

    \section{Personal Contributions}
        My personal contributions towards monitoring and detecting malicious applications that leverage WSL are comprised of a few behavioral
        heuristic algorithms for detecting potentially malicious applications that target WSL, techniques for monitoring Linux applications
        and stopping potentially malicious processes.

        Another personal contribution would be the discovery of a local denial of service attack that leverages an access violation vulnerability
        in lxcore.sys.

        Moreover, I have integrated this SDK in an application in order to exemplify real world usage and integration of the designed
        system. 
        %Moreover, to exemplify the usage and integration of the designed system, I have integrated this SDK in an application
        
    \section{Thesis Outline}
        The thesis can be divided in four main parts. The first part consisting of chapter 2 will contain the current state of WSL and how it can
        be used to run malicious programs, how it can be used to bypass monitoring tools and anti-malware solutions and explain some exploits
        that target WSL.

        The second part, ranging from chapter 3 to chapter 7, presents the technologies and the reasons why they were used in order to
        develop the application as a whole, its architecture along with some implementation details and finally the testing process.

        The third part consists of chapter 7 and will be dedicated to describing the detection algorithms logic and what steps I followed in
        creating them, as well as some implementation details and limitations.

        Lastly, chapter 8 will show possible developement directions for this project, how the current monitoring solution can be improved and
        how it could be extended to achieve more in-depth monitoring and more accurate detection.
        