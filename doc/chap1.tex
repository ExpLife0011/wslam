\chapter{Introduction}
    \section{Motivation}
        \paragraph{} With the release of Windows Subsystem for Linux (WSL), consisting of two new kernel-mode drivers, lxcore.sys and lxss.sys,
        which implement more than  200 linux system calls, a new service, LxssManager, and other user mode components. This opened an attack
        surface that was not covered by monitoring tools untill recent, and doesn't seem to be covered at all by anti-virus vendors.

    \section{Objectives}
        \paragraph{} The main objective is to provide a partial solution to this new security problem. It should defend a system from bashware,
        suspicious interraction between WSL processes and Windows processes and to provide system administrators with enough logs to properly
        respond to incidents, while not disrupting the user experience.

    \section{Personal Contributions}
        \paragraph{}
        My personal contribusions towards monitoring and detecting malicious applications that leverage Windows Subsystem for Linux (WSL) are
        comprised of a few behavioral heuristic algorithms for detecting potentially malicious applications that target WSL, and how these
        applications can be monitored and stopped if malicious. Moreover, I have integrated this monitoring and detection system in a application
        in order to exemplify how easily it could be integrated by any other third party anti-virus vendor.
        
    \section{Thesis Outline}
        \paragraph{}
        Chapter 2 will contain the current state of WSL and how it can be used maliciously ... ceva ceva. The following four chapters will present the
        technologies used in order to develop the application as a whole, its architecture along with some implementation details and finally the
        testing process. Chapter 7 will be dedicated to describing the detection algorithms logic and what steps I followed in creating them, as
        well as some implementation details and limitations. Lastly, chapter 8 will show possible directions for this project, how the current
        monitoring solution can be improved and how it could be extended to achieve more in-depth monitoring and more accurate detection.