\chapter{Possible improvements}
    This being a partial and basic solution to the problem, there are many improvements that can be made, for functional aspects like detection,
    and monitoring capabilities as well as for non-functinoal aspects, like performance, usability and overall robustness of the application.

    \section{Performance}
        There are multiple components that could be further optimized, for example, as of current implementation, the process collector is
        a doubly linked list, which is very inefficient for lookups (O(n) complexity). A more efficient implementation would be a resizable hash
        table using quadratic or lilnear probing. Another example of a component that needs optimization is the dictionary used to match paths
        for which we recheck the security token. A better implementation would use a compressed trie and aho corrasick algorithm for matching
        the path.

        Another performance improvement would be to not rematch the path for the same opened file each time we encounter it, but to cache
        the results in a minifilter define structure called "context", which can be attached to any filter manager object (i.e. file object, 
        volume, instance, etc.).

    \section{Detection}
        Currently, only privilege escalation is covered and, partially, file infectors. A very important type of malware that should be covered
        is ransomware. Ransomware is a type of malware that encrypts files and asks for a ransom, usually in the form of cryptocurency, in order
        to decrypt the user's files.

        \paragraph{}
        Moreover, detection could be improved by integrating next-gen anti-virus features, like AI based detection. Multiple solutions are
        possible and should be investigated, for example Markov Model and Markov Chains or Recurent Neural Networks. These could prove extremely
        efficient because most of the complexity is in the training of the model, while using the model in order to identify malicious actions
        would just follow a simple linear algorithm.

        \paragraph{}
        Finally, in order to more reliably and efficiently detect exploits in general, especially kernel exploits, memory introspection
        techniques could be implemented at hypervisor level. ) Hypervisor perform virtualization without a host operating system between them
        and physicalsystem hardware\cite{Bulazel}. The Xen hypervisor \cite{Barham} has been used in several surveyed analysis systems.
        
        For example, privilege escalation could by detecting by hooking memory access at process security tokens and making use of Extended
        Page Tables (EPT).
        
    \section{Reverting malicious actions}
        In case of behavioral detection, a process could do some malicious actions before being detected. For example, a ransomware may actually
        encrypt some files or parts of some files before being stopped by the anti-virus solution. In this case, it would be needed to store all
        permanent actions (i.e. file delete) and revert them when and if the process is detected.

    \section{Network Filtering}
        A network filter kernel-mode driver is required in order to monitor and report suspicious network activity inside WSL. For example, such
        a component could possibly detect reverse shell behavior.

    \section{User-Mode Hooking Framework}
        In order to obtain more granular monitoring capabilities, a function hooking framework is required. The ability of synchronously monitoring
        specific API usage opens up a range of possibilities, from exploit detection to syscall graph based detection heuristics.
        In order to load a shared library into any Linux pico process, an entry with the path to the library must be added into the ld.so.conf. Since
        any process with root privilleges (inside the linux system) can update the said config file, we must protect it from a kernel driver, so that
        we are sure that our shared library isn't removed. This can be done by filtering IRP\textunderscore MJ\textunderscore WRITE operations on that
        file and readding the entry if removed.
        
        When loaded in a linux process, the library hooks the functions we want to monitor (i.e. ptrace, open) and synchronously notifies a 
        service via local sockets about every hooked function call, informing the service about exploitation intent and failing the function call
        if needed.

        We consider this method unreliable against smarter bashware that actively checks and avoids hooked functions, and possibly completely useless
        starting with Windows RS5 if executable memory can't be allocated anymore.
